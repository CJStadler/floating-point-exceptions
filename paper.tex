\documentclass{article}

\usepackage{listings}
\usepackage[style=numeric]{biblatex}
\addbibresource{references.bib}

\begin{document}

Barr et al. aimed to discover inputs that raise floating point exceptions
in a given program \cite{barr_automatic_2013}. As an extension of this work we
aim to discover inputs that lead to different exception-raising behavior between
two versions of a given program — one optimized with "fast math".

Our approach has two steps:
\begin{enumerate}
  \item Solve: find candidate inputs using an SMT solver (Z3).
  \item Search: Test inputs in the neighborhood of each candidate against P and
P' until one produces different exception traces.
\end{enumerate}

Our first step is conceptually similar to that of Barr et al., differing
only in implementation choices. Our primary contribution is to test the inputs
generated by the first step on both P and P', comparing their exception-raising
behavior.

\section{Find candidate inputs}

To find an input that raises an exception at an operation in a program we should
be able to construct a formula representing the conditions under which an
exception would occur. For example, given an operation \texttt{a + b} we could
construct the formula $(a + b = \infty) \lor (a + b = -\infty)$. Solutions to
this formula would be guaranteed to raise an overflow exception if given as
inputs to the operation. And if the formula is unsatisfiable this would prove
that the operation could not raise the exception.

However, these conclusions are only sound if our formula uses a theory of
floating point numbers which corresponds to the execution environment. SMT
solvers such as Z3 do include floating point theories, but solving these
formulas is not generally practical. Adopting the approach of Barr et al. we
instead construct formulae over the real numbers. To find overflow-raising
inputs for the above example we construct the formula $|a + b| > \Omega$, where
$\Omega$ is the greatest representable floating point value (e.g.
\texttt{DBL\_MAX} if $a$ and $b$ are double precision values). Since floating
point arithmetic is an approximation of real arithmetic we expect that solutions
to this formula are likely to either raise an overflow or be near values which
will raise an overflow.

\begin{table}[h]
\begin{tabular}{lllll}
Expression  & Overflow             & Underflow & Invalid & Divide by Zero \\ \hline
$x \odot y$ & $|x \odot y| > \Omega$ & $0 < |x \odot y| < \omega$ & N/A & N/A \\
$x / y$     & $|x| > |y| \Omega$     & $0 < |x| < |y| \omega $ &
  $x = 0 \land y = 0$ & $x \neq 0 \land y = 0$ \\
\end{tabular}
\caption{Formulae for each exception type. Where $\odot \in \{+, -, *\}$,
  $\Omega$ is the largest representable value, and $\omega$ is the smallest
  positive "normal" value (e.g. \texttt{DBL\_MIN})}
\label{table:formulae}
\end{table}

For each instruction in the given program we construct such a formula for each
type of exception that could be raised (See \ref{table:formulae}). We do not
make formulae for inexact exceptions as these are too common to be notable.

\subsection{Implementation}

Ariadne instruments the program source code with a conditional for each
exception type and then uses KLEE to find inputs that cause the body of the
conditional to be reached \cite[3-4]{barr_automatic_2013}. However, KLEE does
not support floating point values so this approach required modifying KLEE to
interpret floating point values as reals\cite[2]{barr_automatic_2013}.

We instead have implemented our own technique for building these formulae that
does not use KLEE. Our Python program takes two files as inputs: LLVM IR for $P$
and $P'$. Each of these is parsed into an AST using \textit{llvmlite}, which
provides python bindings for LLVM. For each instruction in each program we build
the formulae given by \ref{table:formulae} using Z3.

When an operand of an instruction is an identifier we substitute the
corresponding expression into the formula. For example, to find an input that
causes an overflow to be raised in the second addition below we would solve the
formula $|(a + b) + a| > \Omega$, since $x = a + b$.

\begin{lstlisting}
double add2(double a, double b) {
  double x = a + b;
  double y = x + a;
  return y;
}
\end{lstlisting}

This results in formulae containing only constant terms and the program
parameters, which are left free. A satisfying assignment for such a formula
therefore can be interpreted as inputs to the program.

The union of the sets of formulae from the two programs is then taken. Since $P$
and $P'$ are likely to share many instructions some of the same formulae are
generated from both programs. Finally we solve each formula (using Z3) and
return the satisfying assignments for each satisfiable formula.

\section{Search for exception raising inputs}

The output of the previous step is a set of inputs, each of which we
\textit{expect} to cause an exception in either P or P'. The goal now is to
find inputs that (1) do cause an exception in P or P' and (2) have different
exception-raising behavior in the two programs.

Because our formulae are formulated over the reals but the concrete execution is
done in floating point the candidate inputs may not cause the expected exception
to be raised. For example, suppose for some program that any input $x$ greater
than or equal to some constant $\alpha$ will raise an exception. The solver may
return the assignment $x = \alpha$, but the real value for $x$ may not be
exactly representable in floating point. When given as input to the concrete
program $x$ will therefore need to be rounded to some $x'$. If rounded down then
$x' < \alpha$, and an exception will not be raised.

An insight of Barr et al. was that even if the satisfying assignment does not
raise an exception there is likely to be a floating point number close to it
that does \cite[2]{barr_automatic_2013}. We therefore search the neighborhood of
the satisfying assignment for an exception-raising input. Furthermore, we search
for an input that has different exception-raising behavior in P and P'. If the
optimizations of P' have eliminated the possibility of a certain exception then
we just need to find an input which raises it in P.

Defining "different exception-raising behavior" is difficult because P and P'
represent different computations, even if they are intended to be approximately
the same. For example, suppose for some input both programs raise an overflow
exception in the LLVM instruction \textttt{\%2 = fadd double \%0, \%1}. Despite
having the same representation these instructions could represent different
computations if optimizations have changed the expressions assigned to the
operands. I.e. it is not possible to define a correspondence between
instructions in the two programs.

We have chosen to define "different exception-raising behavior" in terms of the
sequence of exception types produced by each program. If these sequences are
different then we say the programs have different behavior. If they are the
same, regardless of where these exceptions occur in the program, we say they are
not different. This definition is quite strict in what it considers "different".
For example, if both programs produce a single overflow but at very different
points in the program this is not considered a difference. The advantage of this
definition is that there are no "false positives".

\subsection{Implementation}

Since the above definition is in terms of sequences of exceptions we need to
collect these sequences for P and P'. To accomplish this we have implemented an
LLVM pass which adds a call to a \texttt{check_for_exception} function after
every floating point instruction. \texttt{check_for_exception} queries the
floating point status word \cite{noauthor_status_nodate} to determine if an
exception has occurred, and it's type. If one has occurred it records this in a
global variable. After the execution of the program this variable therefore
stores the sequence of exceptions that occurred.

The given program is compiled into P and P', each of these is instrumented using
the pass described above, and these are both linked to a driver program (written
in C++). Candidate inputs are read from a file (produced by the first step).
For each of these inputs we first test it on P and P', and if a difference is
not found we search in the neighborhood of the input. This search is done using
the \texttt{nextafter} function \cite{noauthor_nextafter3_nodate} to find the
next higher and lower representable floating point values. This search is done
until a difference is found or a constant bound is reached. This results in an
exhaustive search centered around the given input.

Because an input to the program may consist of multiple discrete arguments this
search is done for each argument, so that all combinations are tested.

TODO

\section{Results}

\begin{table}[h]
\begin{tabular}{l|llll}
Program                   & Constraints (P, P') & Satisfiable & Unique inputs & Diff producing \\ \hline
turbine1                  & 48 (32, 26)         & 44          & 29            & 14             \\
turbine3                  & 48 (32, 26)         & 44          & 30            & 11             \\
jetengine                 & 106 (58, 58)        & 96          & 41            & 0              \\
carbongas                 & 26 (16, 16)         & 23          & 4             & 0              \\
odometer (2 iterations)   & 190 (122, 76)       & 167         & 70            &
15
\end{tabular}
\end{table}

\section{Algorithm}

\begin{lstlisting}
fun main(source):
  P = compile source to LLVM without optimizations
  P' = compile source to LLVM with optimizations

  inputs = find_inputs(P, P')
  search(inputs, P, P')

fun find_inputs(P, P')
  p_formulae = make_formulae(P)
  p'_formulae = make_formulae(P')

  formulae = p_formulae ∪ p'_formulae

  inputs = collect_inputs(formulae)
  return map ()

fun collect_inputs(formulae):
  solutions = []

  for formula in formulae:
    solution = solve(formula)
    if solution is sat:
      solutions << solution.inputs

  return solutions

fun make_formulae(llvm):
  env = {} # Map of identifiers to symbolic values

  # Require that each input is in the representable range.
  inputs_constraint = True
  for f in llvm.formals:
    env[f] = f
    inputs_constraint = (inputs_constraint and (abs(f) < DBL_MAX))

  formulae = []

  for inst in llvm.instructions:
    result = symbolically_execute(inst, env)
    env[inst.destination] = result

    if inst.op == fdiv:
      # Invalid
      formulae << (inst.numerator == 0 and inst.denominator == 0)
      # DivByZero
      formulae << (inst.numerator != 0 and inst.denominator == 0)
      # Overflow
      formulae << (abs(inst.numerator) > (abs(inst.denominator * DBL_MAX)))
    else:
      # Overflow
      formulae << (abs(result) > DBL_MAX)
      # Underflow
      formulae << (abs(result) > 0 and abs(result) < DBL_MIN)

  formulae = map (f => f and inputs_constraint) formulae
  return formulae

fun search(inputs, P, P'):
  P = instrument P to check for FP exceptions
  P' = instrument P' to check for FP exceptions

  for input in inputs:
    for input in nearby_fp_numbers(input):
      p_trace = exec(P, input)
      p'_trace = exec(P', input)

      if p_trace != p'_trace:
        report_diff(input, p_trace, p'_trace)
\end{lstlisting}

\printbibheading
\printbibliography

\end{document}
